Carto\+Pics is a simple script to generate Digital Elevation Model picture.

 ~\newline
 {\itshape Raster of a Brest\textquotesingle{}s harbour area} 

\subsection*{Getting Started}

These instructions will get you a copy of the project up and running on your local machine for development and testing purposes.

\subsubsection*{Prerequisites}

Let\textquotesingle{}s get started ! Clone the repository using git


\begin{DoxyCode}
git clone https://github.com/Teusner/CartoPics
\end{DoxyCode}


\subsubsection*{Compiling}

You are now able to compile the project


\begin{DoxyCode}
cd CartoPics/src
mkdir build
cd build
cmake ..
make
\end{DoxyCode}


create\+\_\+raster is now generate in Cartopics/src/build

\subsubsection*{First example}

To generate a Digital Elevation Model, you should have a file of the points you measured, as you can see on the following example


\begin{DoxyCode}
# latitude    longitude  elevation
48.29762887 -004.41737340 14.334
48.29762971 -004.41735997 14.379
48.29763698 -004.41738809 14.452
...
\end{DoxyCode}
 And then, run Carto\+Pics with the name of your file in parameter and the number of pixel along the x-\/axis


\begin{DoxyCode}
./cartopics my\_file.txt 1000
\end{DoxyCode}


You can find your output image in the same directory with the name {\itshape my\+\_\+file\+\_\+map.\+ppm}

\subsection*{Built With}


\begin{DoxyItemize}
\item \href{https://github.com/delfrrr/delaunator-cpp}{\tt delaunator-\/cpp} -\/ An efficient delaunay triangulation library
\item \href{https://github.com/richardroberts1992/Spectrum}{\tt Spectrum} -\/ A Color\+Map generator \href{http://cs.swan.ac.uk/~csbob/research/callCenter/color/roberts18spectrum.pdf}{\tt P\+DF Description}
\end{DoxyItemize}

\subsection*{Authors}


\begin{DoxyItemize}
\item {\bfseries Brateau Quentin} -\/ {\itshape Initial work} -\/ \href{https://github.com/Teusner}{\tt Teusner} \+:sunglasses\+:
\end{DoxyItemize}

\subsection*{License}

This project is licensed under the G\+NU General Public License v3.\+0 -\/ see the L\+I\+C\+E\+N\+SE.md file for details 